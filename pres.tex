\documentclass{beamer}
\usetheme[sidebar=false]{uu}
\usepackage{amsmath}
\usepackage{amssymb}
\usepackage{graphicx}

\usepackage{pgf}
\usepackage{tikz}
\usetikzlibrary{arrows, automata}
\usepackage[latin1]{inputenc}

\title[] {Functioneel Programmeren}
\subtitle{in vergelijking met andere paradigmas}
\author {Thomas van Binsbergen}
\date {}

\begin{document}
% new commands
\definecolor{brickred}{HTML}{b6321c}
\newcommand{\mvar}[1]{$\;\texttt{\textcolor{black}{#1}}\;$}
\newcommand{\mkey}[1]{$\;\texttt{\textcolor{orange}{#1}}\;$}
\newcommand{\mmod}[1]{$\;\texttt{\textcolor{purple}{#1}}\;$}
\newcommand{\mdef}[1]{$\;\texttt{\textcolor{blue}{#1}}\;$}
\newcommand{\mcon}[1]{$\;\texttt{\textcolor{brickred}{#1}}\;$}
\newcommand{\mtype}[1]{$\;\texttt{\textcolor{brown}{#1}}\;$}
\newcommand{\mpar}[1]{$(#1)$}
\newcommand{\mnil}[1]{\mcon{[]}}
\newcommand{\cons}[3]{\mpar{\mval{#1}\mcon{#2}\mval{#3}}}
\newcommand{\ra}[0]{$\;\rightarrow\;$}
\newcommand{\Ra}[0]{$\;\Rightarrow\;$}
\newcommand{\ind}[0]{$\;\;\;\;$}

\begin{frame}
    \titlepage
\end{frame}

\begin{frame}
    \frametitle{De paradigma's}
    \begin{itemize}
        \item Imperatief Programmeren (PHP, Javascript, Java, Python, Scala ...)
        \item Object Oriented programmeren (Java, PHP, Scala)
        \item Declaratief/Functioneel Programmeren (Haskell, Scala, ML)
    \end{itemize}
\end{frame}

\begin{frame}
    \frametitle{Programmeren zonder assignment}
    \begin{itemize}
        \item In de wiskunde zijn er veel variaties op het $=$-teken
        \item PHP en Javascript zouden het $:=$ symbool moeten gebruiken
        \item De zogenoemde 'destructieve assignment'
        \item In Haskell bestaat deze operator niet
        \item Daardoor hebben we geen variabelen!
    \end{itemize}
    \includegraphics[width=0.2\textwidth]{no_assignment.png}
\end{frame}

\begin{frame}
    \frametitle{Wat hebben we niet?}
    \begin{itemize}
        \item Variabelen
        \item Arrays (niets anders dan een variabele met structuur)
        \item Objecten (structuren)
        \item For en while loops (om door structuren te reizen)
    \end{itemize}
\end{frame}

\begin{frame}
    \frametitle{Wat hebben we wel?}
    \begin{itemize}
        \item Functies met parameters
        \item Functies zonder parameters, constanten
        \item Datatypes (structuren)
        \item Recursie (om door structuren te reizen)
        \item Sterke (statische) Typering!
    \end{itemize}
\end{frame}

\begin{frame}
    \frametitle{Zuiver programmeren (zonder side-effects)}
    \begin{block}{Voordelen}
        \begin{itemize}
            \item Side-effects vormen een grote bron voor fouten
            \item Eenvoudig redeneren over je programmas
        \end{itemize}
    \end{block}
    \begin{block}{Nadelen}
        \begin{itemize}
            \item Stijlere leercurve
            \item In de praktijk hebben we side-effects:
            \item User-input, databases, file-input, willekeur, parallelisme
        \end{itemize}
    \end{block}
\end{frame}

\begin{frame}
    \frametitle{Statische Typering}
    \begin{block}{Voordelen}
        \begin{itemize}
            \item Veel fouten komen voor 'runtime' aan het licht
            \item Het type van een functie zegt veel over wat de functie doet
            \item Dient als een contract tussen gebruiker en implementeerder van de functie!
        \end{itemize}
    \end{block}
    \begin{block}{Nadelen}
        \begin{itemize}
            \item De wereld van input/output is niet getypeerd, dit maakt parsen/printen noodzakelijk
        \end{itemize}
    \end{block}
    \begin{block}{Dynamische Typering}
        \begin{itemize}
            \item In PHP gebeurd dit parsen/printen achter de schermen 
        \end{itemize}
    \end{block}
\end{frame}

\begin{frame}
    \frametitle{Andere eigenschappen van Haskell}
    \begin{itemize}
        \item Equational Reasoning
        \item Hogere-orde functies (abstractie! abstractie! abstractie!)
        \item Luie evaluatie
        \item Haskell is een uitstekende gastheer voor EDSLs
    \end{itemize}
\end{frame}

\begin{frame}
    \frametitle{Waar wordt Haskell vooral voor gebruikt?}
    \begin{itemize}
        \item Het schrijven van compilers (de ene taal naar de ander omzetten)
        \item Het analyseren en optimaliseren van broncode
        \item Grote financiele applicaties
    \end{itemize}
\end{frame}

\begin{frame}
    \frametitle{Parsen, Transformeren, Printen}
    \begin{block}{De pijpleiding}
        \begin{itemize}
            \item Parsen: Input String $\rightarrow$ Structuur
            \item Transformeren: Structuur  $\rightarrow$ Structuur
            \item Printen: Structuur $\rightarrow$ Output String
        \end{itemize}
    \end{block}
\end{frame}

\end{document}
